
\section{\bf Instructions for the Introduction}

Instructions for creating an introduction for the software manual will be given
in this section. You will need to have an overview of the computer codes written
for this class. You should not discuss details and comparisons of the codes
writteen during the semester. Instead, you should summarize the content
covered in the course and provide general relationships between major content
areas presented in class and the sections in your software manual.

\vskip0.25in\hrule\vskip0.25in

\noindent {\bf Introduction of Software Manual}

The following will be used in assessing the introduction to your software
manual. You should take these as general guidelines and if you have questions
about the introduction to software manual ask your instructor.

\begin{enumerate}
  \item The student demonsstrates a clear understanding of the content in the
        software manual.
    \begin{enumerate}
      \item {\bf Not Addressed:} There is no discussion of the organization of
            the software in the manual. There is no discussion of the purpose of
            the project or the software included in the manual.
      \item {\bf Novice:}  There is a basic summary of the content of the course
            in terms of various codes and routines that have been developed
            during the semester. Maybe something like a chapter by chapter
            summary of what is included in the chapter. No overall puspose for
            the project is included in the introduction.
      \item {\bf Intermediate:} Discussion of relationships between software in
            the various chapters is included in the introduction along with a
            brief summary of the purpose of the project is included.
      \item {\bf Expert:} Examples defining relationships bewtween chapters and
            software routines is included. A thorough discussion of the purpose
            of the project as it relates this and other classes and possible
            work is included in the introduction.
    \end{enumerate}
\end{enumerate}

\noindent To get the highest marks on this project you should aim for the expert
level. Please note that the introduction should be no more than a few pages. Be
concise in your comments and overview.

\newpage
